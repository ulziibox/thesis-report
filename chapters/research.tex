\chapter{Сэдвийн ерөнхий судалгаа}

Уг сэдвийн судалгаанд 

\section{Үндсэн ойлголтууд}

\subsection{Sub section 1}

Lorem ipsum

\section{Ижил төсөөтэй систем}

Lorem ipsum

\subsection{Sub section 2}

\section{Ашиглах технологи}

\subsection{React - Javascript сан}

Фэйсбүүк компани дотооддоо ашиглаж байсан технологио 2013 онд танилцуулсан нь програмчлалын Javascript хэлийг ашиглаж хийсэн Front-end library болох React\footnote{Reactjs official site \url{https://reactjs.org}} технологи юм. Declarative UI хөгжүүлэлтийн аргыг хамгийн анх дэлгэрүүлж, өргөн хэрэглээнд нэвтрүүлж чадсан тул Declarative UI-н гол төлөөлөгч гэж явдаг. Уг технологийг ашиглахын тулд үндсэн хэдэн ойлголтууд авах хэрэгтэй. Үүнд component ба түүний lifecycle, javascript-н өргөжүүлсэн хувилбар болох jsx, мөн хамгийн чухал зүйл болох Virtual DOM нар багтана. 

Declarative UI гэдэг нь хэрэглэгчийн интерфэйсийн кодыг бичихдээ юу зурагдах буюу render хийх үеийн интерфэйсийг бүгдийг урьдчилан тодорхойлдог арга барил юм. Imperative програмчлалаас ялгаатай нь хязгаартай нөхцөлд яг юу хийхийг хатуугаар зааж өгөхгүйгээр тухайн state-с хамааруулж хэрэглэгчийн хүссэн зүйлийг гаргаж өгөх боломжтой.   

React нь component-based буюу DOM дээр хэвлэж байгаа бүх зүйлс component байна гэсэн дүрмийг баримталдаг. Component үүсгэж бичихийн давуу тал нь нэг бичсэн кодоо олон дахин бичигдэхээс зайлсхийж, дахин ашиглах боломжийг олгодог. Тус бур өөрсдийн гэсэн дотоод төлөвтэй мөн гаднаас утга хүлээн авах чадвартай. Үүнийг бид Props гэж нэрлэдэг. Мөн component нь stateless, stateful гэж хоёр хуваагддаг ба stateful component нь өөрийн гэсэн төлөвтэй, түүнийгээ удирддаг, class болон hook ашигласан функцууд байна. React-н давуу тал нь state эсвэл props-н өөрчлөлтийг үргэлж хянаж байдаг тул өөрчлөлт орж ирэхэд бүтэн хуудсыг зурах бус зөвхөн тухайн өөрчлөгдсөн component-г л дахин зурдаг. Ингэснээр энгийн вэбүүдээс илүү хурдтай ажилладаг.

JSX нь Javascript Extended гэсэн үгний товчлол бөгөөд энгийнээр javascript дотор HTML-н тагуудыг бичиж өгөх мөн кодыг илүү богино болгож хүссэн үр дүндээ хүрэх боломжийг олгодог. Үүний цаана Babel гэсэн transcompiler-г ашиглаж дундын хөрвүүлэлтийг хийдэг ба хэдийгээр HTML таг бичиж байгаа харагддаг ч код дунд цэвэр HTML-г огтоос бичиж өгдөггүй гэсэн үг юм.

\begin{lstlisting}[language=Javascript, caption=JSX ашиглаж 'container' класстай html элемент буцаах компонент, frame=single]
	export function Container = ({children}) => {
		return (
			<div className="container">
				{children}
			</div>
		)
	}
			
\end{lstlisting}

Жинхэнэ DOM дээр богино хугацаанд олон өөрчлөлт хийхэд удах асуудал гарсан тул React маань Virtual DOM гэсэн хийсвэр давхарга үүсгэж өөрчлөлтүүдээ Virtual DOM дээрээ хадгалаад нэгдсэн нэг өөрчлөлтийг жинхэнэ DOM руугаа дамжуулдаг. 

\subsection{Next.js - React дээр суурилсан фрэймворк}

Next.js\footnote{Next.js official site \url{https://nextjs.org}} нь React сан дээр суурилж хөгжүүлсэн нээлттэй эхийн фрэймворк бөгөөд Vercel компани 2016 онд албан ёсны танилцуулгаа хийж олон нийтэд зарласан юм. React нь зөвхөн хэрэглэгчийн интерфэйсийг зурах үүрэгтэй сан ба бусад вэб хөгжүүлэлтэд хэрэгтэй хуудас хооронд шилжих гэх мэт үйлдлийг react-router болон бусад маш олон нэмэлт сангаас сонголт хийж шийдэх шаардлагатай байсан нь төслийн эхлэх явцыг удаашруулах хандлагатай байдаг. Харин Next.js ашигласнаар нэг ч тохиргоо хийлгүйгээр төслийг эхлүүлж шууд код бичих боломжийг бүрдүүлдэг. Цаана нь хийгдсэн тохиргоо нь нийт вэбсайтуудын 90 хувийн шаардлагыг хангаж чаддаг гэж үздэг нь уг фрэймворкын сүүлийн жилүүдэд эрэлттэй болж буй шалтгаануудыг нэг билээ. Иймд хэрэглэгчийн харагдах хэсгийг Next.js фрэймворк дээр хийх нь хамгийн тохиромжтой гэж үзэж байна.

Next.js давуу талуудаас дурьдвал:
\begin{itemize}
	\item Image Optimization буюу их хэмжээтэй зураг оруулахад автоматаар зургийн чанарыг алдагдалгүйгээр хэмжээг багасгаж өгдөг
	\item Zero config буюу нэг ч тохиргоо хийлгүйгээр төслөө эхлүүлэх боломж
	\item Static Site Generator болон Server Side Render хийх
	\item Typescript болон Fast Refresh дэмждэг
	\item File-system Routing буюу “pages” гэсэн хавтас дотор үүссэн файлуудаас хамаарч вэбийн замууд тодорхойлогддог мөн dynamic routing ашиглах боломжтой
	\item API Routes буюу өөр дээрээ nodejs сервер ашиглаж API endpoint гаргах боломжтой. Ингэснээр тусдаа сервер ашиглах шаардлага үүсэхгүй 
	\item SEO буюу хайлтын системийн оновчлолыг SSR ашиглаж тохируулж өгөх гэх мэт маш олон давуу талуудтай
\end{itemize}

Мөн ердөө ганц “build” коммандаар статик болон динамик вэбийг гарган авч ямар нэгэн вэб сервер /apache, nginx гэх мэт/ ашиглалгүйгээр сервер дээрээ шууд байршуулах боломжтой юм.

\subsection{PostgreSQL - Өгөгдлийн сан}

\subsection{Prisma - Нээлттэй эхийн ORM}

\subsection{Figma - Интерфэйсийн дизайн гаргах багаж}

